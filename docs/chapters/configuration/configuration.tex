\LevelZero{Configuration}\label{ch:configuration}


\LevelOne{Main-Control File}

\LevelTwo{Configuration}
\begin{Configuration}
    \item[core]{
        The \ttt{core} options inside \ttt{config/fuji/config.json} will influence \ttt{all modules}.

        \begin{NestedList}
            \item[debug]{
                \item[]

                \begin{NestedList}
                    \item[disable\_all\_modules]{
                        Used to test the compatibility between fuji and other mods.
                    }

                \end{NestedList}

            }
        \end{NestedList}

        \begin{NestedList}
            \item[backup]{
                Fuji will back up the \ttt{config/fuji} directory automatically before it loads any module.

                \begin{NestedList}
                    \item[max\_slots]{
                        How many \ttt{backup} should we keep?
                    }

                    \item[skip]{
                        The list of \ttt{path resolver} to skip in backup. \\
                        Insert \ttt{modules/head} means skip the folder \ttt{config/fuji/modules/head}.
                    }
                \end{NestedList}
            }
        \end{NestedList}


        \begin{NestedList}
            \item[language]{
                \item[]

                \begin{NestedList}
                    \item[default\_language]{
                        The default language to use.

                        \begin{tips}{Enable multi-language support for fuji}
                            See~\nameref{sec:language}
                        \end{tips}
                    }

                \end{NestedList}

            }
        \end{NestedList}

        \begin{NestedList}
            \item[quartz]{
                Fuji use \ttt{quartz} library as scheduler, all the~\nameref{sec:job} are managed by quartz.
                Quartz library use a language called \ttt{cron language} to define when to trigger a job.

                \begin{NestedList}
                    \item[logger\_level]
                    The logger level for \ttt{quartz}.
                    The logger level from high to low are: OFF, FATAL, ERROR, WARN, INFO, DEBUG, TRACE, ALL.

                    \begin{example}{Enable all logs for quartz}
                        Set the value to \ttt{ALL} to display all the messages from quartz.
                        It's recommended to set at least \ttt{WARN} level, to avoid \ttt{console spam}.
                    \end{example}

                \end{NestedList}

            }
        \end{NestedList}


    }

\end{Configuration}

\clearpage
\LevelOne{Module-Control File}
You can read more about \ttt{module-control file} for each \ttt{module} in \nameref{ch:module}
